\section{Konzeption}

\subsection{Festlegung der Systemanforderungen}

Die grundlegende Anforderung an diese Studienarbeit war es, ein Sprecheridentifikationssystem zu entwickeln, welches über eine Webapplikation bedient werden kann.
Da diese Anforderungsbeschreibung noch viel Spielraum über den Funktionsumfang dieses Systems lässt, werden in diesem Kapitel die Anforderungen an das System genauer spezifiziert.


\begin{itemize}
    \item Das System soll lediglich ein Demo-System darstellen, welches die Machbarkeit eines text-unabhängigen Sprecheridentifikationssystems aufzeigt. Es soll nicht in einem produktiven Umfeld eingesetzt werden können.
    \item Der Datensatz für das System ist zu Begin festgelegt. Das heißt, es werden nur bereits bekannte Sprecher identifiziert. Zudem können keine neuen Sprecher registriert werden, wodurch das Sprecheridentifikationssystem keine dynamische Erweiterung um Sprecher bereitstellen muss.
    \item Das System beschränkt sich auf das Identifizieren eines Sprechers unter 20 Sprechern.
    \item Zur Authentifizierung eines Sprechers wird ein 20 sekündiger Audio-Clip an das System übermittelt. Der Audio-Clip ist dem System bisher unbekannt, aber in gleichen Bedingungen aufgenommen, wie die bekannten Clips.
    \item Das Sprecheridentifikationssystem sollte Teil des Back-Ends sein. Die Bedienung erfolgt über eine Webapplikation, die mit dem System über eine Schnittstelle kommuniziert.
    \item In der Web-Oberfläche des Demo-Systems soll der zu authentifizierende Sprecher (einer der 20) und ein Verifikations-Clip ausgewählt werden können. Es soll für jeden Sprecher 5 Clips zur Auswahl geben.
    \item So soll der Nutzer des Demo-Systems testen können, was passiert, wenn ein zu dem zu authentifizierenden Sprecher passender bzw. unpassender Verifikations-Clip ins System gegeben wird.
    \item Das System soll dem Nutzer dann Informationen über die Identifikations-Verteilung (zu welchem Sprecher passt der Clip zu welchem Prozentsatz) und den Authentifizierungs-Status (\gqq{erfolgreich}/\gqq{nicht erfolgreich}) als Rückmeldung darstellen.
\end{itemize}

\subsection{Konzept Versuchssystem}

\subsubsection{Systemidee}

Aus der Literaturrecherche in Kapitel \ref{stand_der_technik} gehen verschiedene Stimmmerkmale zur Benutzerauthentifizierung hervor.
Die Ergebnisse der dargestellten Untersuchungen unterscheiden sich, in wie weit die unterschiedlichen Stimmerkmale die Stimme repräsentieren bzw. zuverlässig für eine korrekte Authentifizierung sind.

Da die Features \ac{MFCC}, \ac{LPC}, \ac{LPCC}, \ac{MFCC} und \ac{dMFCC} die grobe Schnittmenge der Untersuchungen darstellt, werden diese in einer eigens durchgeführten Versuchsreihe getestet und evaluiert.
Hierfür wird ein System entworfen, das die Merkmale aus einem Datenset extrahiert und verschiedene Kombinationen vergleicht.
Das Ziel dieses Systems ist, eine ideale Kombination aus Features herrauszufinden.

\subsubsection{Feature Kombinationen}

Um die Kombinationen zu generieren, werden zunächst die verschiedenen Möglichkeiten definiert und anschließend miteinander kombiniert, was zu über 500 verschiedenen Konfigurationen führt.

\begin{lstlisting}[language=JavaScript,numbers=none,caption=Konfigurationsmöglichkeiten,label=configs]
Configs = {
    "amount_of_frames": [10000, 15000],
    "size_of_frame": [400, 600],
    "LPC": {
        "order": [13, 20],
        "weight": [0, 1]
    },
    "MFCC": {
        "order": [13, 20],
        "weight": [0, 1]
    },
    "LPCC": {
        "order": [13, 20],
        "weight": [0, 1]
    },
    "delta_MFCC": {
        "order": [13],
        "weight": [0, 1]
    }
}
\end{lstlisting}

Die Werte für die Anzahl und Länge der Frames \codestyle{amount_of_frames} und \codestyle{size_of_frame} berechnen sich aus vorherigen Versuchen, der verwendeten Abtastrate und der gewünschten Trainingsclip-Länge.


Bei den Features gibt der \codestyle{order} Parameter die Anzahl der Features pro Frame an.
Der Wert 13 ist aus der Literatur abgeleitet. %TODO CITE
Der Wert 20 ist selbstständig ergänzt, um einen Vergleich zu ermöglichen.
Um die Anzahl der Konfigurationen zu reduzieren und da die Relevanz von \ac{dMFCC} am geringsten ist, wird hier auf den zweiten Wert verzichtet.
Der \codestyle{weight} Parameter gibt lediglich an ob in dieser Konfiguration dieses Feature verwendet werden soll oder nicht.
% Dazu werden in einem ersten Schritt Konfigurationen definiert, welche die unterschiedlichen Zusammensetzungen der Features beschreibt.
% Anschließend werden alles diese Zusammensetzungen berechnet und durch neuronale Netze klassifiziert.
% Abschließend erfolgt eine Auswertung der verschiedenen Konfigurationen um eine ideale Konfiguration von Features zu ermitteln
