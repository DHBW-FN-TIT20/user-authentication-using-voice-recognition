\section{Umsetzung}

\subsection{Technologieentscheidung}
\subsubsection{Versuchssystem}
Aus der Konzeption in Kapitel~\ref{sec:Konzeption} gehen verschiedene Anforderungen an die Technologie hervor.
Durch die Verwendung von neuronalen Netzen werden die verfügbaren Technologien bzw. Programmiersprachen bereits eingeschränkt.
Um weitere Technologieenscheidungen zu treffen muss zunächst eine Programmiersprache ausgewählt werden.
Da in einer Umfrage unter Machine-Learning-Entwicklern und Data Scientists 57\% Python als Programmiersprache nutzen und 33\% diese sogar priorisieren, wird für dieses Projekt Python als Programmiersprache festgelegt \autocite[vgl. ][S. 16]{vision_mobile_state_2017}.
In einer allgemeinen Umfrage zu verwendeten Programmiersprachen liegt Python mit 49,2\% auf Platz drei \autocite[vgl.][]{yepis_2023_2023}.
Das bedeutet das knapp die Hälfte der befragten Entwickler unter anderem Python verwenden, somit ist eine ausreichende Verbreitung gewährleistet

Um Machine-Learning mit Python zu betreiben, gibt es mehrere Bibliotheken, die zur Auswahl stehen und nachfolgend verglichen werden.


\subsubsection{Demosystem}

\subsection{Versuchssystem}
\subsection{Demosystem}