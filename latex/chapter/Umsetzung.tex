\section{Umsetzung}

\subsection{Technologieentscheidung}
\subsubsection{Versuchssystem}
Aus der Konzeption in Kapitel~\ref{sec:Konzeption} gehen verschiedene Anforderungen an die Technologie hervor.
Durch die Verwendung von neuronalen Netzen werden die verfügbaren Technologien bzw. Programmiersprachen bereits eingeschränkt.
Um weitere Technologieenscheidungen zu treffen muss zunächst eine Programmiersprache ausgewählt werden.
Da in einer Umfrage unter Machine-Learning-Entwicklern und Data Scientists 57\% Python als Programmiersprache nutzen und 33\% diese sogar priorisieren, wird für dieses Projekt Python als Programmiersprache festgelegt \autocite[vgl. ][S. 16]{vision_mobile_state_2017}.
In einer allgemeinen Umfrage zu verwendeten Programmiersprachen liegt Python mit 49,2\% auf Platz drei \autocite[vgl.][]{yepis_2023_2023}.
Das bedeutet, dass knapp die Hälfte der befragten Entwickler unter anderem Python verwenden, somit ist eine ausreichende Verbreitung gewährleistet.
Die Verbreitung ist vorallem deshalb ein wichtiger Aspekt, da es durch eine hohe Verbreitung ein großes Eco-System mit viele quelloffene Ressourcen und bereits durchgeführte Projekte gibt.

Um Machine-Learning mit Python zu betreiben, gibt es mehrere Bibliotheken zur Auswahl.
Die 3 bekanntesten Open-Source-Bibliotheken sind dabei \gqq{TensorFlow}, \gqq{PyTorch} und \gqq{SciKit-Lern} \autocite[vgl.][]{msv_tensorflow_nodate}.

Diese Bibliotheken sind sich relativ ähnlich.
Vorteile einer Bibliothek bringen automatisch auch Nachteile, die sich somit gegenseitig aufheben.

Der wesentliche Unterschied zwischen den Bibliotheken ist, dass die Entwickler dieser Studienarbeit bereits Vorkenntnisse in Tensorflow haben, weshalb Tensorflow verwendet wird.
Zusätzlich wird das Modul \gqq{Keras} verwendet, welches eine entwicklerfreundliche High-Level-Schnittstelle für Tensorflow ist und somit eine einfachere Entwicklung ermöglicht \autocite[vgl.][]{noauthor_keras_nodate}.

\subsubsection{Demosystem}

\subsection{Versuchssystem}

\subsubsection{Feature Kombination}
Wie bereits in Kapitel~\ref{sec:FeatureKombination} erwähnt müssen zuerst die Konfigurationen erzeugt werden.
Hierzu werden die vorher definierten Werte in einer JSON Datei erfasst und durch ein Tool werden alle möglichen Konfiguration mit einer ID erzeugt.
Hierbei muss beachtet werden, dass durch die Kombination Konfigurationen entstehen in welchen, nichts berechnet werden muss.
Die erstellte JSON Datei ist in dem Listing~\ref{configs} dargestellt.
\begin{lstlisting}[language=JavaScript,numbers=none,caption=Konfigurationsmöglichkeiten,label=configs]
Configs = {
    "amount_of_frames": [10000, 15000],
    "size_of_frame": [400, 600],
    "LPC": {
        "order": [13, 20],
        "weight": [0, 1]
    },
    "MFCC": {
        "order": [13, 20],
        "weight": [0, 1]
    },
    "LPCC": {
        "order": [13, 20],
        "weight": [0, 1]
    },
    "delta_MFCC": {
        "order": [13],
        "weight": [0, 1]
    }
}
\end{lstlisting}
Der \codestyle{weight} Parameter gibt lediglich an ob in dieser Konfiguration dieses Feature verwendet werden soll oder nicht.


\subsubsection{Datensatz}

\subsection{Demosystem}