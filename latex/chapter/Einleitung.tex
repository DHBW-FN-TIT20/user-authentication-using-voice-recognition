\section{Einleitung}
\subsection{Problemstellung und Relevanz}
In der heutigen digitalen Welt, in der Sicherheit und der Schutz personenbezogener Daten von entscheidender Bedeutung sind und viele Prozesse online und digital abgewickelt werden, gewinnt die Benutzerauthentifizierung als zentraler Faktor zunehmend an Bedeutung.
Traditionelle Authentifizierungsmethoden, wie Passwörter oder PINs haben ihre Grenzen und können durch technologische Fortschritte und kreative Angriffsstrategien oft überwunden werden.
Um den wachsenden Anforderungen der Authentifizierung gerecht zu werden, haben Forscher und Entwickler alternative Ansätze erforscht.
\newparagraph
Ein vielversprechender entstandener Ansatz ist die Authentifizierung eines Benutzers durch seine Stimmmerkmale.
Die Stimme ist wie zum Beispiel der Fingerabdruck ein eindeutiges biometrisches Merkmal.
Jeder Mensch besitzt eine einzigartige Kombination von Stimmmerkmalen, welche es ermöglichen somit einen Nutzer eindeutig zu identifizieren.
Dieser Ansatz bietet die Möglichkeit einer natürlichen und bequemen Authentifizierung, da die meisten Menschen über ein funktionierendes Sprachorgan verfügen.

\subsection{Zielsetzung}

\subsection{Aufbau der Arbeit}