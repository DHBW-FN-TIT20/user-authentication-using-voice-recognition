\section{Einleitung}
\subsection{Problemstellung und Relevanz}
In der heutigen digitalen Welt, in der Sicherheit und der Schutz personenbezogener Daten von entscheidender Bedeutung sind und viele Prozesse online und digital abgewickelt werden, gewinnt die Benutzerauthentifizierung als zentraler Faktor zunehmend an Bedeutung.
Traditionelle Authentifizierungsmethoden, wie Passwörter oder PINs haben ihre Grenzen und können durch technologische Fortschritte und kreative Angriffsstrategien oft überwunden werden.
Um den wachsenden Anforderungen der Authentifizierung gerecht zu werden, haben Forscher und Entwickler alternative Ansätze erforscht.

Ein vielversprechender entstandener Ansatz ist die Authentifizierung eines Benutzers durch seine Stimmmerkmale.
Die Stimme ist wie zum Beispiel der Fingerabdruck ein eindeutiges biometrisches Merkmal.
Jeder Mensch besitzt eine einzigartige Kombination von Stimmmerkmalen, welche es ermöglichen somit einen Nutzer eindeutig zu identifizieren.
Dieser Ansatz bietet die Möglichkeit einer natürlichen und bequemen Authentifizierung, da die meisten Menschen über ein funktionierendes Sprachorgan verfügen.

\subsection{Zielsetzung}\label{sec:Zielsetzung}
Im Rahmen dieser Studienarbeit wird untersucht wie zuverlässig und wie sicher die Authentifizierung eines Benutzers über seine Sprachmerkmale realisierbar ist.
Dafür sollen die zugrunde liegenden Technologien und Verfahren zur Stimmerkennung erörtert werden und verschiedene Ansätze zur Benutzerauthentifizierung anhand der Stimmmerkmale beleuchtet und verglichen werden.
Auf Grundlage dieser Recherche soll ein Sprecherauthentifikationssystem entwickelt werden.
Dieses Systems soll in eine einfache Web-Anwendung integriert werden, dabei soll der Authentifizierungsprozess des Sprecherkennungssystems serverseitig angebunden werden.
Abschließend sollen die potenziellen Sicherheitsrisiken des Systems betrachtet und bewertet werden.

Die Erkenntnisse dieser Arbeit zeigen den aktuellen Stand der Forschung und Entwicklung im Bereich der Benutzerauthentifizierung mittels Stimmmerkmalen auf.
Durch das bessere Verständnis dieser Technologie können potenzielle Anwendungsfälle und mögliche Innovationen identifiziert werden.

\subsection{Aufbau der Arbeit}
Der Aufbau der vorliegenden Studienarbeit gliedert sich in mehrere Abschnitte.
Zu Beginn werden die Grundlagen in Bezug zur Benutzerauthentifizierung mittels Stimmmerkmalen erläutert, dabei werden die grundlegenden Algorithmen und Verfahren zur Extraktion von Stimmmerkmalen dargestellt.
In einem Folgeschritt wird der Stand der Technik beleuchtet, hierzu werden mehrere wissenschaftliche Arbeiten aus anderen Forschungen verglichen.
In weiteren Schritt wird die Konzeption sowohl für das Versuchssystem als auch das Demosystem erstellt und umgesetzt.
Anschließend werden die Ergebnisse des Versuchssystems evaluiert.
In einem letzten Schritt werden Sicherheitsrisiken von Sprecherauthentifikationssystemen analysiert und bewertet.
Die Arbeit endet mit einer kritischen Reflexion, einem Fazit und einem Ausblick.