\section{Stand der Technik} \label{sec:StandDerTechnik}

\textauthor{\vJB}{}{}

Im vorliegenden Kapitel wird der aktuelle Stand im Bereich der Sprechererkkenung detailliert untersucht und dargestellt.
Dazu werden mehrere Arbeiten und Bücher aus anderen wissenschaftlichen Forschungen verglichen.

Für die Analyse von Sprachsignalen werden zunächst Merkmale (Features) des Signales benötigt, welche dann weiterverarbeitet werden können.
Für die Weiterverarbeitung wird dann ein Klassifikations-Model benötigt, das aufgrund der Merkmale eine Aussage über den Sprecher treffen kann.

In der vorhandenen Literatur finden sich mehrere Bücher und Fachartikel, die sich mit diesem Thema bereits auseinandergesetzt haben. 
Dieses Kapitel zeigt, welche Merkmale von verschiedene vorangehende Arbeiten verwendet werden und wie effektiv diese im direkten Vergleich sind. Dabei wird ebenfalls auf die verwendeten Klassifikationsmodelle eingegangen.

Qi (Peter) Li, ein Elektrotechnik-Doktor und Experte für Sprecherauthentifikation, untersucht in seinem Buch \gqq{Speaker Authentification}, relevante Merkmale für die Sprechererkennung \autocite{li_speaker_2012}.

Er gelangte zu folgenden Erkenntnissen:
\ac{MFCC} und \ac{LPC} sind sehr gute Merkmale zur Identifizierung eines Sprechers in ruhigen Umgebungen, aber ihre Leistung nimmt in Umgebungen mit Hintergrundrauschen ab \autocite[vgl.][S. 136]{li_speaker_2012}.
Die ersten und zweiten Ableitungen von \ac{MFCC} werden oft auch noch genutzt, tragen laut Qi (Peter) Li allerdings nicht zu einer Verbesserung des Authentifizierungssystems bei \autocite[vgl.][S. 143]{li_speaker_2012}.

Neben den Hauptmerkmalen \ac{MFCC} und \ac{LPC} werden zusätzlich Gammatone-Merkmale bewertet, die das menschliche Hören modellieren sollen.
Diese können auch relevante Daten für eine Sprecherauthentifikation liefern \autocite[vgl.][S. 111, 117]{li_speaker_2012}.

Qi (Peter) Li hat im Zuge seiner Arbeit auch ein ganz neues Merkmal (CFCC\footnote{Cochlear filter cepstral coefficients}) entwickelt, das ebenfalls das menschliche Gehör nachahmen soll und in seinen Untersuchungen am besten abgeschnitten hat \autocite[vgl.][S. 135]{li_speaker_2012}.

Das Buch \gqq{Advanced Topics In Biometrics} von Haizhou Li, Kar-ann Toh und Liyuan Li bezeichnet \ac{MFCC} als das relevanteste Merkmal für die Sprecherauthentifikation, obwohl es ursprünglich für die Spracherkennung entwickelt wurde \autocite[vgl.][S. 7, 51]{li_advanced_2011}.
Aber auch \ac{LPCC}, PLP\footnote{Perceptual Linear Prediction} und FM\footnote{Frequency Modulation} werden als effektive Merkmale genannt und können zusammen mit \ac{MFCC} sehr gute Ergebnisse erzielen \autocite[vgl.][S. 6, 67]{li_advanced_2011}.

Das Paper \gqq{Multilingual Speaker Recognition Using Neural Network} protokolliert gute Ergebnisse durch die Extraktion von \ac{LPC} und der Nutzung eines neuronalen Netzes \autocite[vgl.][S. 9]{kumar_rajeev_multilingual_2009}.

Auch eine ältere Untersuchung von Reynolds and Rose ergab, dass \ac{LPC} in einer ruhigen Umgebung effektiv ist, \ac{MFCC} allerdings robuster ist und daher von den Autoren bevorzugt wurde \autocite[vgl.][S. 2f]{reynolds_robust_1995}.
Als Klassifikations-Model wurden hierbei Gaussian Mixture Models (GMM) genutzt und als effektiv erachtet.
Auch neuronale Netze und Hidden Markov Modelle werden als mögliche Klassifikatoren erwähnt, allerdings nicht getestet \autocite[vgl.][S. 2f, 11]{reynolds_robust_1995}.

Ein weiters Paper, welches sich mit Sprecherauthentifikation auf performance-kritischen Systemen beschäftigt, nennt \ac{LPCC} als bestes Merkmal \autocite[vgl.][S. 7]{thullier_text-independent_2017}.
Hierbei wurde ein Naive-Bayes-Klassifikator verwendet, welcher gute Ergebnisse erzielte \autocite[vgl.][S. 18f]{thullier_text-independent_2017}.

Fatma Zohra Chelali1 und Amar Djeradi geben in ihrem Artikel zur Sprechererkennung zunächst \ac{MFCC} und PLP als relevante Merkmale an \autocite[vgl.][S. 276]{chelali_text_2017}.
Allerdings vergleichen sie auch verschiedene Kombinationen von \ac{MFCC}, den Ableitungen \ac{dMFCC} und \ac{ddMFCC}, DWT\footnote{Discrete Wavelet Transform} und \ac{LPC}.
Bei diesen Untersuchungen hat die Kombination von \ac{MFCC}, \ac{dMFCC}, DWT und \ac{LPC} am besten abgeschnitten.
\ac{ddMFCC} konnte das System nicht verbessern \autocite[vgl.][S. 276, 739]{chelali_text_2017}.
Hier wurde ein neuronales Netz als Klassifikations-Model verwendet \autocite[vgl.][S. 735]{chelali_text_2017}.

Ein Paper der Tokyo Institute of Technology hatte ebenfalls mit \ac{MFCC} und \ac{LPC} sehr gute Ergebnisse.
Durch das Hinzufügen von ZCR\footnote{Zero-crossing Rate} konnten diese noch verbessert werden.
Die Versuche wurden hier mit einem neuronalen Netz und einer Support-Vector-Machine durchgeführt.
Die beiden Klassifizierer haben sich dabei in ihrer Effektivität kaum unterschieden, mit dem Neuralen Netz wurden allerdings die besten Ergebnisse erzielt \autocite[vgl.][S. 4]{neha_chauhan_2019_2019}.
\begin{landscape}
    \begin{figure}[p]
        \centering
        \includegraphics[width=1.4\textwidth, keepaspectratio]{images/vergleich-feature-extraction.pdf}
        \caption{Literaturbewertung zu Features}
        \label{fig:vergleichFeatureExtraction}
    \end{figure}
\end{landscape}

Die Abbildung~\ref{fig:vergleichFeatureExtraction} zeigt nochmals eine tabellarische Übersicht der verschiedenen Merkmale, die in den analysierten Büchern und Artikeln verwendet wurden und ihre bewertete Effektivität.

Die analysierten Bücher und Artikel legen nahe, dass eine Kombination von \ac{MFCC}, \ac{LPC} und \ac{LPCC} sinnvoll für eine Sprecherauthentifikation ist.
Zwar hatten einige Paper noch vereinzelt andere Features als effektiv erachtet, die grobe Schnittmenge besteht jedoch aus den oben genannten Features.

Als Klassifikationsmodelle wurden in den meisten Experimenten neuronale Netze verwendet, aber auch GMM, HMN und SVM wurden als effektiv erachtet.