\section{Stand der Technik}

Für die Analyse von Sprachsignalen werden zunächst Merkmale (Features) des Signales benötigt, welche dann weiterverarbeitet werden können.
Für die Weiterverarbeitung wird dann ein Klassifikations-Model benötigt, das aufgrund der Merkmale eine Aussage über den Sprecher treffen kann.

In der vorhandenen Literatur finden sich mehrere Bücher und Fachartikel, die sich mit diesem Thema bereits auseinandergesetzt haben. 
Dieses Kapitel zeigt, wie verschiedene vorrangeganene Projekte Features erhoben haben, welche Merkmale davon als effektiv erachtet wurden und welche Klassifikations-Modelle genutzt werden können.

Qi (Peter) Li, ein Elektrotechnik-Doktor und Experte für Sprecherauthentifizierung, untersuchte in seinem Buch \gqq{Speaker Authentification}, relevante Merkmale für die Sprechererkennung \autocite{li_speaker_2012}.

Er gelangte zu folgenden Erkenntnissen:
MFCC und LPC sind sehr gute Merkmale zur Identifizierung eines Sprechers in ruhigen Umgebungen, aber ihre Leistung nimmt in Umgebungen mit Hintergrundrauschen ab \autocite[vgl.][S. 136]{li_speaker_2012}.
Die ersten und zweiten Ableitungen von MFCC werden oft auch noch genutzt, tragen laut QI (Peter) Li allerdings nicht zu einer Verbesserung des Authentifizierungssystems bei \autocite[vgl.][S. 143]{li_speaker_2012}.

Neben den Hauptmerkmalen MFCC und LPC wurden in dem Buch auch Gammatone-Merkmale bewertet, die das menschliche Hören modellieren sollen.
Diese können auch relevante Daten für eine Sprecherauthentifizierung liefer \autocite[vgl.][S. 111, 117]{li_speaker_2012}.

Qi (Peter) Li hat im Zuge seiner Arbeit auch ein ganz neues Merkmal (CFCC) entwickelt, das ebenfalls das menschliche Gehör nachahmen soll und in seinen Untersuchungen am besten abgeschnitten hat \autocite[vgl.][S. 135]{li_speaker_2012}.

Das Buch \gqq{Advanced Topics In Biometrics} von Haizhou Li, Kar-ann Toh und Liyuan Li bezeichnet MFCC als das relevanteste Merkmal für die Sprecherauthentifizierung, obwohl es ursprünglich für die Spracherkennung entwickelt wurde \autocite[vgl.][S. 7, 51]{li_advanced_nodate}.
Aber auch LPCC, PLP und FM werden als effektive Merkmale genannt und können zusammen mit MFCC sehr gute Ergebnisse erzielen \autocite[vgl.][S. 6, 67]{li_advanced_nodate}.

Das Paper \gqq{Multilingual Speaker Recognition Using Neural Network} protokoliert gute Ergebnisse durch die Extraktion von LPC und der Nutzung eines neuronalen Netzes \autocite[vgl.][S. 9]{kumar_rajeev_multilingual_nodate}.

Auch eine ältere Untersuchung von Reynolds and Rose ergab, dass LPC in einer ruhigen Umgebung effektiv ist, MFCC allerdings robuster ist und daher von den Autoren bevorzugt wurde \autocite[vgl.][S. 2f]{reynolds_robust_1995}.
Als Klassifikations-Model wurden hierbei Gaussian Mixture Models (GMM) genutzt und als effektiv erachtet.
Allerdings werden auch neuronale Netze, als ein neueres Modell, aufgeführt und auch Hidden Markov Models, welche eigentlich für Spracherkennung genutzt werden, werden als Möglichkeit erwähnt \autocite[vgl.][S. 2f, 11]{reynolds_robust_1995}.

Ein anderes Paper, bei dem die benötigte Rechenleistung für die Merkmalsberechnung kritisch war, nannte LPCC als bestes Merkmal \autocite[vgl.][S. 7]{thullier_text-independent_2017}.
Hierbei wurde ein Naive-Bayes-Klassifikator verwendet, welcher gute Ergebnisse erzielte \autocite[vgl.][S. 18f]{thullier_text-independent_2017}.

Fatma Zohra Chelali1 und Amar Djeradi geben in ihrem Artikel zur Sprechererkennung zunächst MFCC und PLP als relevante Merkmale an \autocite[vgl.][S. 276]{chelali_text_2017}.
Allerdings vergleichen sie auch verschiedene Kombinationen von MFCC, dMFCC, ddMFCC, DWT und LPC.
Bei diesen Untersuchungen hat die Kombination von MFCC, dMFCC, DWT und LPC am besten abgeschnitten.
ddMFCC konnte das System nicht verbessern \autocite[vgl.][S. 276, 739]{chelali_text_2017}.
Hier wurde ein neuronales Netz als Klassifikations-Model verwendet \autocite[vgl.][S. 735]{chelali_text_2017}.

Ein Paper der Tokyo Institute of Technology hatte ebenfalls mit MFCC und LPC sehr gute Ergebnisse.
Durch das Hinzufügen von ZCR konnten diese sogar noch verbessert werden.
Die Versuche wurden hier mit einem neuronalen Netz und einer Support Vector Machine durchgeführt.
Die beiden Klassifizierer haben sich dabei in ihrer Effektivität kaum unterschieden, mit dem Neuralen Netz wurden allerdings die besten Ergebnisse erziehlt \autocite[vgl.][S. 4]{neha_chauhan_2019_2019}.

Die analysierten Bücher und Artikel legen nahe, dass eine Kombination von MFCC, LPC und LPCC sinnvoll für eine Sprecherauthentifizierung ist.
Zwar hatten einige Paper noch vereinzelt andere Features als effektiv erachtet, die grobe Schnittmenge besteht jedoch aus den oben genannten Features.

Als Klassifikations-Modelle wurden in den meisten Experimenten neuronale Netze verwendet, aber auch GMM, HMN und SVM wurden als effektiv erachtet.