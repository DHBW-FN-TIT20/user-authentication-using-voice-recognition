\section{Stand der Technik}

\subsection{Feature Extraction}

Für die Analyse von Sprachsignalen werden zunächst Merkmale (Features) des Signales benötigt, welche dann weiterverarbeitet werden können...
Dieses Kapitel zeigt, wie verschiedene vorrangeganene Projekte Features erhoben haben...

In der Literatur finden sich mehrere Bücher und Fachartikel, die sich mit diesem Bereich bereits auseinandergesetzt haben. 

Qi (Peter) Li, ein Elektrotechnik-Doktor und Spezialist im Bereich Sprecherauthentifizierung, untersuchte für sein Buch \gqq{Speaker Authentification}, welche Merkmale für eine Sprechererkennung relevant sind \autocite{li_speaker_2012}.

Dabei kam er auf folgende Erkenntnisse:
MFCC und LPC sind in ruhigen Umgebungen sehr gute Merkmale, um einen Sprecher zu identifizieren, werden allerdings in Umgebungen mit Hintergrundrauschen schlechter \autocite[vgl.][S. 136]{li_speaker_2012}.
Die ersten und zweiten Ableitungen von MFCC werden oft auch noch genutzt, tragen laut QI (Peter) Li allerdings nicht zu einer Verbesserung des Authentifizierungssystems bei \autocite[vgl.][S. 143]{li_speaker_2012}.

Neben den Hauptfeatures MFCC und LPC wurden in dem Buch auch noch Gammatone Features, welche das menschliche Hören modellieren sollen, bewertet.
Diese können auch relevante Daten für eine Sprecherauthentifizierung liefer \autocite[vgl.][S. 111, 117]{li_speaker_2012}.

Qi (Peter) Li hat im Zuge seiner Arbeit auch noch ein ganz neues Merkmal (CFCC) entwickelt, das ebenfalls das menschliche Gehör nachahmen soll und in seinen Untersuchungen am besten abgeschnitten hat \autocite[vgl.][S. 135]{li_speaker_2012}.

Das Buch \gqq{Advanced Topics In Biometrics} von Haizhou Li, Kar-ann Toh und Liyuan Li nennt MFCC das relevanteste Feature für die Sprecherauthentifizierung, auch wenn das Merkmal uhrsprünglich für die Spracherkennung entwickelt wurde \autocite[vgl.][S. 7, 51]{li_advanced_nodate}.
Aber auch LPCC, PLP und FM wurden als effektiv bezeichnet und können zusammen mit MFCC sehr gute Ergebnisse erziehlen \autocite[vgl.][S. 6, 67]{li_advanced_nodate}.

Das Paper \gqq{Multilingual Speaker Recognition Using Neural Network} hatte mit der Extraktion von LPC und der Nutzung eines Neuronalen Netzes gute Ergebnisse erziehlt \autocite[vgl.][S. 9]{kumar_rajeev_multilingual_nodate}.

Auch eine ältere Untersuchung von Reynolds and Rose ergab, dass LPC in einer ruhigen Umgebung effektiv ist, MFCC allerdings robuster ist und daher von den Autoren bevorzugt wurde \autocite[vgl.][S. 2f]{reynolds_robust_1995}.

Ein anderes Paper, bei welchem die für die Merkmalberechnung benötigte Computing-Power kritisch war, nannte LPCC als bestes Merkmal \autocite[vgl.][S. 7]{thullier_text-independent_2017}.

Fatma Zohra Chelali1 und Amar Djeradi gaben in ihrem Artikel zur Sprechererkennung zunächst MFCC und PLP als relevante Merkmale an \autocite[vgl.][S. 276]{chelali_text_2017}.
Allerdings vergleichen sie auch verschiedene Kombinationen von MFCC, dMFCC, ddMFCC, DWT und LPC.
Bei diesen Untersuchungen hat die Kombination von MFCC, dMFCC, DWT und LPC am besten abgeschnitten.
ddMFCC konnte das System nicht verbessern \autocite[vgl.][S. 276, 739]{chelali_text_2017}.

Ein Paper der Tokyo Institute of Technology hatte ebenfalls mit MFCC und LPC sehr gute Ergebnisse.
Durch das Hinzufügen von ZCR konnten diese sogar noch verbessert werden \autocite[vgl.][S. 4]{neha_chauhan_2019_2019}


Die analysierten Bücher und Artikel lassen darauf schließen, dass eine Kombination von MFCC, LPC und LPCC sinnvoll für eine Sprecherauthentifizierung ist.
Zwar hatten einige Paper noch vereinzelt andere Features als effektiv erachtet, die grobe Schnittmenge besteht jedoch aus den oben genannten Features.