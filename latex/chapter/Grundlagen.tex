\section{Grundlagen}
\subsection{Künstliche Intelligenz}
\ac{KI} hat in den letzten Jahren stark Aufmerksamkeit gewonnen und gilt als eine der wichtigsten Technologien des 21. Jahrhunderts.
Spätestens seit der Veröffentlichung und dem Erfolg durch den Chatbot ChatGPT am 30. November 2022 sind die Möglichkeiten von \ac{KI} den meisten Menschen bekannt. ChatGPT hat innerhalb von fünf Tagen die Schwelle von einer Million Nutzer erreicht. Spotify oder Facebook zum Vergleich haben hierfür 5 beziehungsweise 10 Monate benötigt \autocite[vgl. ][]{janson_infografik_2023}.
\newparagraph
Die Statistik die im Jahre 2022 vom IDC erstellt wurde prognostiziert für die Anwendungsfelder Hardware, Software und IT-Services im Jahre 2024 einen weltweiten Umsatz von 554,3 Milliarden US-Dollar.
Dies entspricht einer Steigerung von 171 Milliarden US-Dollar (44,6 \%) im Vergleich zum Jahr 2021 \autocite[vgl. ][]{idc_kunstliche_2022}.
Dies unterstreicht noch einmal die Relevanz von \ac{KI} im aktuellen Zeitalter.
Doch was ist \ac{KI} und wie ist Sie für diese Arbeit von Relevanz, diese Fragen werden im Folgenden beantwortet.
% ---- %
\newparagraph
Die Idee einer \acl{KI} ist bereits seit Mitte des 20. Jahrhunderts existent. 1955 definierte John McCarthy, einer der Pioniere der \ac{KI}, als erster den Begriff der \acl{KI} wie folgt:
\begin{quote}
  \gqq{Ziel der KI ist es, Maschinen zu entwickeln, die sich verhalten, als verfügten sie über Intelligenz.}
  \autocite[][S. 1]{ertel_grundkurs_2016}
\end{quote}\noindent
Aus heutiger Sicht ist diese Definition nicht mehr ausreichend.
Eine weit verbreitete Definition stammt von Elaine Rich die sagt, dass sich \ac{KI} mit der Entwicklung von Computeralgorithmen befasse, für Probleme, die Menschen im Moment noch besser lösen können \autocite[vgl. ][]{rich_artificial_1983}.
%Themenfelder KI