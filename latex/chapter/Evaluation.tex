\section{Evaluation} \label{sec:Evaluation}
% TODO durchführung ??
In dem vorliegenden Kapitel wird die Evaluierung der Ergebnisse des Versuchssystems durchgeführt, hierzu werden die Ergebnisse aufbereitet und anhand mehrerer Metriken verglichen.

Für die Auswertung der Zuordnungsverteilungen aller Konfigurationen des Versuchssystems werden die Ergebnisse zunächst grafisch dargestellt.
Dabei werden in allen Grafiken die Ergebnisse der drei neuronalen Netze pro Konfiguration miteinander Verrechnet.
Es werden vier verschiedene Aspekte betrachtet (vgl. Abbildung~\ref{fig:AuswertungVersuchssystem}).
\begin{figure}[H]
    \centering
    \includegraphics[width=1\textwidth, keepaspectratio]{images/Auswertung.png}
    \caption{Auswertung Versuchssystem}
    \label{fig:AuswertungVersuchssystem}
\end{figure}

Für die durchschnittliche absolute Genauigkeit (erstes Diagramm) werden die Wahrscheinlichkeiten für den zu identifizierenden Benutzer pro Testdatei aufsummiert und der Durchschnitt gebildet.
Eine Genauigkeit von 80 \% entspricht damit der Aussage, dass das neuronale Netz in Kombination mit der jeweiligen Konfiguration dem zu authentifizierenden Benutzer im Durchschnitt 80 \% der Frames der Testdatei zuordnet.

% TODO: Eingehen auf Closed-Set sobald die Entscheidung im Konzept beschrieben ist (Threshold)
Bereits in dieser Grafik zeigt sich die Bildung von Clustern.
Die besten Konfigurationen erreichen eine Genauigkeit von über 65 \%, weshalb dieser Wert für die folgenden drei Auswertungen verwendet wird.
Hier werden die Testdateien mittels dieses Wertes in die drei Kategorien korrekt zugeteilt, falsch zugeteilt und nicht zugeteilt eingeordnet.

Eine Testdatei gilt als korrekt zugeteilt, wenn dem zu authentifizierenden Benutzer mindestens 65 \% der Frames zugeordnet werden.
Analog gilt die Testdatei als nicht korrekt zugeteilt, wenn einem anderen Benutzer mindestens 65 \% der Frames zugeordnet werden.
Erreicht kein Nutzer mindestens 65 \%, so wird die Testdatei als nicht zugeteilt eingestuft.

Unter Betrachtung der durchschnittlichen absoluten Genauigkeit, sowie der korrekt zugeteilten Testdateien (zweites Diagramm), ergeben sich die vier markierten Cluster als beste Konfigurationen.
Auch in den zwei verbleibenden Kategorien, zeichnen sich diese Konfigurationen vor allem durch eine niedrige Anzahl an nicht zugeordneten Dateien (kleiner 25 \%, viertes Diagramm), sowie falsch zugeordneter Dateien (kleiner 2 \%, drittes Diagramm) aus.

Die Bildung der Cluster ist dabei auf die Art und Weise wie die Konfigurationen erzeugt werden zurückzuführen.
Da hier eine bestimmte Systematik vorliegt, enthalten diese Cluster jeweils ähnliche Feature-Kombinationen, wobei die für die Ergebnisse relevanten Features in jeder Konfiguration des Clusters vorhanden sind.
In den rot markierten Clustern sind dies \ac{MFCC} und \ac{dMFCC} Features.

In einer detaillierteren Analyse im Direktvergleich ergeben sich die in Tabelle~\ref{table:ergebnisOutput} dargestellten Konfigurationen als beste Kombinationen:
\begin{table}[H]
    \centering
    \begin{tabular}{c|c|c|c|c|c}
    ID  & Durchschnittliche absolute Genauigkeit & \ac{LPC} & \ac{MFCC} & \ac{LPCC} & \ac{dMFCC} \\ \hline
    472 &                                 0.8033 &        0 &        20 &         0 &         13 \\ \hline
    474 &                                 0.8015 &        0 &        20 &        13 &         13 \\ \hline
    410 &                                 0.8015 &        0 &        20 &        13 &         13 \\ \hline
    408 &                                 0.7989 &        0 &        20 &         0 &         13 \\ \hline
    414 &                                 0.7986 &        0 &        20 &        13 &         13 \\
    \end{tabular}
    \caption{Auswertung der Konfigurationen}
    \label{table:ergebnisOutput}
\end{table}

Die Evaluation ergibt somit, dass die Konfiguration mit der ID 472 die besten Ergebnisse erzielt.
Dabei werden 1500 Frames bei einer Frame-Größe von 600 Samples generiert.
Daraufhin werden 20 \ac{MFCC} Koeffizienten, sowie 13 \ac{dMFCC} Koeffizienten pro Frame erzeugt, welche durch das neuronale Netz ausgewertet werden.
\ac{LPC} und \ac{LPCC} zeigen in den ausgewählten Konfigurationen keinen signifikanten Mehrwert, weshalb diese nicht verwendet werden.

In einem weiteren Schritt werden basierend auf der Konfiguration 472 weitere Konfigurationen erstellt, um weitere Untersuchungen durchzuführen.
Dabei werden die Parameter Anzahl der Frames, Länge der Frames, sowie Anzahl der \ac{MFCC} Koeffizienten um jeweils einen neuen Wert erweitert, da hier zu erkenen war, dass eine Erhöhung dieser Werte zu einer Verbesserung des Gesamtergebnisses führt.
Folgend wurden die Parameterwerte 20000 Frames, 800 Samples pro Frame und 27 \ac{MFCC} Koeffizienten evaluiert.
Die Parameterverteilung der Konfigurationen sind in Tabelle~\ref{table:additionalKonfigs} dargestellt.
\begin{table}[H]
    \centering
    \begin{tabular}{c|c|c|c|c}
    ID  & Anzahl Frames & Länge Frames & \ac{MFCC} & \ac{dMFCC} \\ \hline
    511 & 20000         & 600          & 20        & 13     \\ \hline
    512 & 15000         & 800          & 20        & 13     \\ \hline
    513 & 15000         & 600          & 27        & 13     \\ \hline
    514 & 20000         & 800          & 20        & 13     \\ \hline
    515 & 20000         & 600          & 27        & 13     \\ \hline
    516 & 15000         & 800          & 27        & 13    
    \end{tabular}
    \caption{Zusätzliche Konfigurationen}
    \label{table:additionalKonfigs}
\end{table}


Die vorliegenden Konfigurationen wurden mithilfe desselben Verfahrens ausgewertet, wie die Konfigurationen zuvor.
Die Untersuchungen der Ergebnisse dieser Konfigurationen zeigen minimale Verbesserungen, die auf die erhöhte Datenmenge zurückzuführen sind.
Da diese Verbesserungen im Verhältnis zur steigenden Rechenzeit minimal sind, werden diese nicht weiter verfolgt.
Somit steht die Konfiguration \textbf{472} aus dem vorherigen Durchlauf als finale Kombination fest.

% TODO Auswertung einfügen.