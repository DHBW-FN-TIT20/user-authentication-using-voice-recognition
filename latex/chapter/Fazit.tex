\section{Fazit und kritische Reflexion}
Im Rahmen dieser Arbeit wurde ein Versuchssystem, sowie ein drauf aufbauendes Demosystem zunächst konzipiert und anschließend implementiert.
Ausgehend von einer ausführlichen Grundlagenrecherche wurden dabei zunächst die Features \ac{LPC}, \ac{LPCC}, \ac{MFCC} und \ac{dMFCC} in Kombination mit einem \ac{NN} als geeignete Kandidaten für die Umsetzung einer Sprecherauthentifizierung ermittelt.
In einem weiteren Schritt ergab sich daraus sowohl die grundlegende Architektur für ein Sprecherauthentifizierungssystem, als auch eine spezifische Architektur für das Versuchs- und Demosystem.
Dazu wurden zunächst Systemanforderungen definiert, um die zu entwickelnden Systeme genauer einzugrenzen und grundlegende Entscheidungen zu treffen.
In einer ausführlichen Technologieentscheidung, wurden verschiedene Technologien miteinander verglichen und schlussendlich eine begründete Wahl getroffen.
Diese bildetet die Grundlage für die anschließende Implementierung der beiden Systeme.
Die Ergebnisse der Durchführung des Versuchssystems sind in er Evaluation ausführlich beschrieben und führen zu der Schlussfolgerung, dass \ac{MFCC} und \ac{dMFCC} für die Sprecherauthentifizierung relevant sind, während \ac{LPC} und \ac{LPCC} keine gewinnbringenden Ergebnisse erzielen.
Die ermittelte Konfiguration zum Einsatz im Demosystem authentifiziert dabei in 90 \% der Fälle die richtige, in 10 \% keine und in 0 \% die falsche Person.
Aufbauend auf diesem Ergebnis, wurde abschließend eine Sicherheitsbewertung in Form einer Threat-Analyse durchgeführt, welche hauptsächlich verfahrensspezifische Sicherheitsrisiken auflistet und bewertet.
Insgesamt konnte somit erfolgreich gezeigt werden, dass eine Sprecherauthentifizierung innerhalb einer fest definierten Gruppe an Benutzern nicht nur möglich, sondern auch zuverlässig umsetzbar ist.
Die abschließende Sicherheitsbewertung zeigt dabei auf, dass Sprecherauthentifizierung aufgrund der vielen verschiedenen Risiken nicht sicher eingesetzt werden kann.
Durch die zusätzliche Integration des Authentifizierungsprozesses in das Demosystem, wurde das in Kapitel~\ref{sec:Zielsetzung} definierte Ziel dieser Arbeit somit erreicht.

% Kritische Punkte
% - Auswahl/verwendeter Datensatz -> gleiche Qualität, ein Redefluss, nur Englisch, nur 20 Personen
% - Neuronales Netz ist fix definiert -> Erfahrungswert, keine ausführliche Literaturbegründung
% - Abhängigkeit bei Erzeugung der NN zwar durch 3 Netze pro Konfig abgefangen, jedoch nicht 100% vergleichbar/reproduzierbar