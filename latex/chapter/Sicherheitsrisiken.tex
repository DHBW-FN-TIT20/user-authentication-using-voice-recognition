% TODO Auch mögliche Lösungen werden vorgestellt nicht nur die Risiken.
\section{Analyse der Sicherheitsrisiken}
In dem vorliegenden Kapitel wird die Analyse der Sicherheitsrisiken von Sprecherauthentifikationssystemen durchgeführt.
Dabei wird die Analyse zunächst auf ein allgemeines System bezogen und die erarbeiteten Sicherheitsrisiken dann auf das entwickelte Demosystem bezogen.
Dies ist möglich da das Demosystem, wessen Entwicklungen in den vorausgegangenen Kapiteln beschrieben wurde, von diesem allgemeinen System abgeleitet ist.

\subsection{Allgemeine Sicherheitsrisiken}
In diesem Abschnitt erfolgt die Analyse der Sicherheitsrisiken für ein allgemeines Sprecherauthentifikationssystem.
Ein solches System wird in Kapitel~\ref{sec:allgemeiner_system_aufbau} dargestellt und erläutert.
Grundsätzlich gibt es drei Arten von Sicherheitsrisiken für Sprecherauthentifizierungssysteme: Voice-Spoofing, Backdoor-Attacken und Eavesdropping.
Im Folgenden sind diese drei Sicherheitsrisiken dargestellt und erläutert.

Voice Spoofing bezieht sich auf die Nachahmung einer bestimmten Stimme oder die Manipulation einer Sprachaufnahme um so jemanden zu täuschen oder Zugriff auf ein System zu erhalten.
Die Nachahmung der Stimme wird auch als Stimmimitation bezeichnet eine Untersuchung der Universität of Birmingham in Alabama zeigt, dass bereits wenige Minuten Audio des Opfers ausreichen um die Stimme vollständig zu klonen.
Auch die Optionen zur Beschaffung dieser Informationen werden präsentiert.
Der Angreifer kann sich in der Nähe des Opfers aufhalten und Sprachaufnahmen machen, im Internet nach Aufnahmen suchen oder gezielt über Spam-Anrufe die benötigten Daten erhalten \autocite[vgl.][]{katherine_shonesy_uab_2015}.


















































































